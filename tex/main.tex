\documentclass[]{article}

% Basic language setup
\usepackage[utf8]{inputenc}
\usepackage[T1]{fontenc}
\usepackage{lmodern}
\usepackage[UKenglish]{babel}
\usepackage[autostyle]{csquotes}
\usepackage[en-GB]{datetime2}

\usepackage{
    graphicx,
    subfiles
}
\usepackage[
    paper=a4paper,
    margin=2.5cm,
    headheight=14pt,
]{geometry}
\usepackage{pdflscape}
\usepackage{booktabs}
\usepackage{multicol}
\usepackage{mathtools, amssymb, amsmath}
\usepackage{afterpage}
\usepackage{graphbox}
\usepackage[noend]{algpseudocode}
\usepackage{algorithm}

% Use paragraph as subsubsubsection
\usepackage{titlesec}
\setcounter{secnumdepth}{4}
\titleformat{\paragraph}
{\normalfont\normalsize\bfseries}{\theparagraph}{1em}{}
\titlespacing*{\paragraph}
{0pt}{3.25ex plus 1ex minus .2ex}{1.5ex plus .2ex}

\usepackage[
  sorting=nty,
  backend=bibtex,
  giveninits=true,
  style=authoryear-comp,
  natbib=true,
  maxcitenames=2,
  maxbibnames=99,
  useprefix=true,
  uniquename=init
]{biblatex}

\addbibresource{bibliography.bib}
% Prioritize DOI > Eprint > URL
\AtEveryBibitem{%
  \ifboolexpr{ test {\ifentrytype{article}} or test {\ifentrytype{book}} or test {\ifentrytype{inproceedings}} }{%
    \iffieldundef{doi}{%
      \iffieldundef{eprint}{}{%
        \clearfield{url} % Keep eprint if present, remove URL
      }
    }{%
      \clearfield{eprint} % Remove eprint if DOI exists
      \clearfield{url}    % Remove URL if DOI exists
    }%
  }{}%
}
\newcommand{\todo}[1]{{\color{red}[\textit{TODO: #1}]}}
\newcommand{\todonocomment}[1]{{\color{red}[\textit{#1}]}}


% Make abstract part of page numbering
\makeatletter
\newif\if@abstractmode

\renewenvironment{titlepage}
{
    \if@twocolumn
    \@restonecoltrue\onecolumn
    \else
    \@restonecolfalse\newpage
    \fi
    \if@abstractmode
    \thispagestyle{plain}%
    \stepcounter{page}%
    \else
    \thispagestyle{empty}%
    \setcounter{page}\@ne%
    \fi
}%
{\if@restonecol\twocolumn \else \newpage \fi
    \if@twoside\else
    \if@abstractmode
    \else
    \setcounter{page}\@ne%
    \fi
    \fi
}

\AtBeginEnvironment{abstract}{%
    \@abstractmodetrue%
}

\makeatother

% * Title page constants
\makeatletter
\newcommand{\@studentnumber}{}
\newcommand{\@firstsupervisor}{}
\newcommand{\@secondsupervisor}{}
\newcommand{\@dailysupervisor}{}
\newcommand{\@externalsupervisor}{}

\renewcommand{\title}[1]{
    \renewcommand{\@title}{#1}
}

\renewcommand{\author}[2]{
    \renewcommand{\@author}{#1}
    \renewcommand{\@studentnumber}{#2}
}
\newcommand{\supervisors}[4]{
    \renewcommand{\@firstsupervisor}{#1}
    \renewcommand{\@secondsupervisor}{#2}
    \renewcommand{\@dailysupervisor}{#3}
    \renewcommand{\@externalsupervisor}{#4}
}
\makeatother

% * Title page setup

% Command to make the lines in the title page
\newcommand{\HRule}{\rule{.9\linewidth}{.6pt}}


\usepackage{tikz}
\usetikzlibrary{shapes.geometric, arrows}

\tikzstyle{process} = [rectangle, 
minimum width=12cm, 
minimum height=1cm, 
text width=11cm, 
draw=black, 
fill=white,
rounded corners]

\tikzstyle{subprocess} = [rectangle, 
minimum width=5cm, 
minimum height=0.6cm, 
text centered, 
text width=5cm, 
draw=black, 
fill=white]

\tikzstyle{arrow} = [thick,->,>=stealth]

\begin{document}
\title{Integrated Vehicle and Crew Scheduling for Electric Buses}
\author
    {T.D. \textsc{van der Plas}} % Name
    {7077270}           % Student number
\date{\today}

\supervisors
    {Dr.~J.A. \textsc{Hoogeveen}} % First reader
    {Dr.~M.E. \textsc{van Kooten Niekerk}} % Second reader
    {MSc.~P. \textsc{de Bruin}} % Second reader
    {MSc.~M. \textsc{Stikvoort}} % Second reader

\subfile{titlepage.tex}

\section{Introduction}
Electric vehicles are beginning to make up large portions of the fleet for public transport providers. In The Netherlands for example, approximately 21\% of all registered buses utilize an electric drivetrain, as reported by the \citet{RDW}. To comply with strict Dutch and \citet{europaRegulation20181999} on climate and sustainability, this proportion must increase substantially; from 2025 onward, the procurement of new buses powered by non-renewable energy sources will no longer be permitted, with the objective that by 2030, all public transport buses operate with zero emissions. Individual line operators such as \citet{qbuzzQbuzz} are therefore rapidly replacing their remaining buses that use fossil fuels, resulting in primarily electric-powered fleets.

Effective use of these new electric vehicles is however more challenging than that of a combustion based fleet. One of the logistical steps in which this is most apparent is that of vehicle scheduling. The Vehicle Scheduling Problem (VSP) aims to assign vehicles such that a set of trips are driven at minimum cost; in the case of buses, these trips are defined by the timetables for individual lines. A typical solution to this problem comes in the form of a collection of vehicle tasks, each of which can be seen as a schedule that an individual bus will follow throughout the day. It may start at a bus storage facility (more commonly called a depot), then perform one or more trips, before finally returning to the depot. In order to string multiple trips together, a bus must make an empty travel from the ending location of one trip to the starting location of the next. These empty travels, often called deadheads, are the main focus when determining the overall costs of a collection of vehicle tasks. There is ample choice as to which deadheads can be driven, as in theory the deadhead between each pair of compatible trips can be included during the creation of a vehicle task. It is therefore necessary to determine which of these deadheads are required in order to drive all trips while minimizing costs.

In the VSP and its variants, a commonly made assumption is that a vehicle can travel an entire day without having to refuel. With electric vehicles, this assumption is often not valid: recharging throughout the day is required as the range of an electric vehicle is limited. Additionally, recharging takes a long time when compared to the refueling of a combustion vehicle: refueling can be done in a matter of minutes, whereas recharging may take multiple hours for a full charge. The electrification of fleets therefore requires the VSP to be adapted to include maximum range constraints, as well as charging possibilities within a vehicle task. This extended version of the problem, often referred to as the Electric Vehicle Scheduling Problem (E-VSP), is significantly harder than its non-electric counterpart; the VSP with a single depot can be solved in polynomial time, whereas the E-VSP is NP-Hard as shown by \citet{Bunte2009} and \citet{Sassi2014} respectively.

As self-driving vehicles are not in general use yet at time of writing, buses also require a driver in order to be operated. The Crew Scheduling Problem (CSP) is therefore commonly the step that follows vehicle scheduling; in this, the objective is to make an assignment of drivers such that a driver is always present when a vehicle is moving. A driver may drive multiple vehicles throughout the day, handing vehicles over to other drivers at designated relief locations. Driver shifts consisting of driving time and breaks are constructed while following local labor regulations. The CSP is also NP-Hard, as shown by \citet{Fischetti1989}.

Costs in the crew scheduling step often exceed those associated with the vehicles themselves, with an estimate by \citet{Perumal2019Crew} putting crew member costs at around 60\% of overall operational costs. Optimal solutions for the CSP are directly influenced by the selected vehicle tasks; depending on which deadhead is selected, the amount of vehicle idle time before a trip may vary. It may therefore not always be feasible for a driver to perform a handover or break between two trips depending on the deadheads used in the vehicle tasks. Sequentially minimizing costs in the VSP and CSP may therefore not result in a solution with overall lowest costs, as the crew schedules in the CSP may be improved greatly by incurring higher driving costs in the VSP. This was already pointed out in the 1980s by \citet{Bodin1983}, who instead advocated for an integrated approach; here, the costs of the VSP and CSP are minimized simultaneously, resulting in the Vehicle and Crew Scheduling Problem (VCSP). \\

A lot of work has already been done on the VSP, CSP and VCSP. Both the sequential and integrated approach have been extensively studied since the 1980s, as shown by reviews such as \citet{Ibarra-Rojas2015} and \citet{Ge2024}. However as mentioned before, the introduction of electric vehicles has introduced significant constraints on recharging times and vehicle ranges, invalidating the critical assumption in many of these works that a vehicle was able to drive an entire day without being refueled. In response, the E-VSP has also been the focus of many studies going back to around 2014. We refer the reader to a survey by \citet{Perumal2022LitRev} for a detailed overview of recent progress.

The integrated VCSP with electric vehicles (E-VCSP) has seen less attention; to the best of our knowledge, only 5 works consider this problem. In these, simplifying assumptions are made about battery charging behavior which might limit real world applicability or accurate modeling of costs. In this work, our aim is therefore to introduce an integrated E-VCSP model which incorporates more realistic behavior for battery charging.

The rest of this work is organized as follows. In Section 2, we will give more background information on the E-VCSP and provide a formal problem definition. In Section 3, we review literature related to the E-VCSP and identify gaps in the current research. In Section 4, we discuss the methodology used in order to solve the E-VCSP. \dots. An overview of nomenclature and abbreviations used throughout this work has been included in Table \ref{tab:nomenclature}.

\begin{table}
  \centering
  \begin{tabular}{ll}
    \toprule
    \multicolumn{1}{l}{\textbf{Abbreviation}} & \multicolumn{1}{l}{\textbf{Definition}}               \\
    \cmidrule(lr){1-1}\cmidrule(lr){2-2}
    ALNS                                      & Adaptive Large Neighborhood Search                    \\
    B\&P                                      & Branch-and-Price                                      \\
    CG                                        & Column Generation                                     \\
    CP                                        & Constraint Programming                                \\
    CSP                                       & Crew Scheduling Problem                               \\
    E-\dots                                   & Problem \dots with electric vehicles                  \\
    LNS                                       & Large Neighborhood Search                             \\
    LS                                        & Local Search                                          \\
    MDVSP                                     & Multi Depot Vehicle Scheduling Problem                \\
    MIP                                       & Mixed Integer Program                                 \\
    SAA                                       & Simulated Annealing Algorithm                         \\
    SDVSP                                     & Single Depot Vehicle Scheduling Problem               \\
    SoC                                       & State of Charge                                       \\
    TCO                                       & Total Cost of Ownership                               \\
    ToU                                       & Time of Use                                           \\
    TVSP                                      & Integrated Timetabling and Vehicle Scheduling Problem \\
    VCSP                                      & Integrated Vehicle and Crew Scheduling Problem        \\
    VSP                                       & Vehicle Scheduling Problem                            \\
    \bottomrule
  \end{tabular}
  \caption{Nomenclature used in this work}
  \label{tab:nomenclature}
\end{table}

\section{Problem definition}
\label{sec:problem_def}
In this section, we give a definition of the integrated Electric Vehicle and Crew Scheduling Problem (E-VCSP). In order to do so, we will first give a global overview of the problem at hand. Afterwards, we discuss the data which is given, before finally formalizing the resulting problem.

\begin{figure}
  \centering
  \begin{tikzpicture}[node distance=2cm]
    \node (strategic) [process, minimum height=2.3cm] {\textbf{Strategic planning}\\\textit{Long term}};
    \node at (strategic.base) (infra) [subprocess, xshift=2.8cm, yshift=0.4cm] {Infrastructure};
    \node at (strategic.base) (line) [subprocess, below of=infra, yshift=1cm] {Line planning and frequency};

    \node (tactical) [process, minimum height=4.5cm, below of=strategic, yshift=-2cm] {\textbf{Tactical planning}\\\textit{Medium-Long term}};
    \node at (tactical.base) (timetable) [subprocess, xshift=2.8cm, yshift=1.5cm] {Timetabling};
    \node at (tactical.base) (vehicle) [subprocess, below of=timetable, yshift=1cm] {Vehicle scheduling (VSP)};
    \node at (tactical.base) (crew) [subprocess, below of=vehicle, yshift=1cm] {Crew scheduling (CSP)};
    \node at (tactical.base) (VCSP) [rounded corners, draw=black, fit=(vehicle) (crew), align=left] {\hspace{-4em}VCSP};
    \node at (tactical.base) (rostering) [subprocess, below of=crew, yshift=1cm] {Crew rostering};

    \node (operational) [process, minimum height=1.3cm, below of=tactical, yshift=-1.5cm] {\textbf{Operational planning}\\\textit{Day of operations}};
    \node at (operational.base) (recovery) [subprocess, xshift=2.8cm, yshift=-0.1cm] {Recovery};

    \draw [arrow] (infra) -- (line);
    \draw [arrow] (line) -- (timetable);
    \draw [arrow] (timetable) -- (vehicle);
    \draw [arrow] (vehicle) -- (crew);
    \draw [arrow] (crew) -- (rostering);
    \draw [arrow] (rostering) -- (recovery);
  \end{tikzpicture}
  \caption{A general overview of the public transport planning process, based on \citet{Ceder1986, Ibarra-Rojas2015, Perumal2022LitRev}.}
  \label{fig:planning-overview}
\end{figure}

\subsection{Overview}
The E-VCSP is part of the planning process used in bus public transit, as can be seen in Figure \ref{fig:planning-overview}. It comes directly after the timetabling step, in which trips are defined for each line according to passenger demand. Each individual timetabled trip represents a bus traveling along a line at a specified time. Oftentimes, these trips are scheduled regularly throughout the day from the start to the end of a line, however irregular schedules and partial travels are also possible.

With these trips being laid out, we now need to determine how they will be driven. Individual trips often have lengths of around 0.5-2 hours, therefore making it possible to perform multiple trips with a single vehicle and driver in a day. In order to perform a trip, a vehicle must travel to its starting location before driving it; this travel is called a deadhead. Deadheads occur between trips and between a trip and a depot. During a deadhead, the vehicle itself is empty except for the driver. It is therefore beneficial to minimize the total amount of time that vehicles and crew members spend driving deadheads, as costs incurred during this time do not have any direct benefit to passengers.

When considering electric vehicles, the time between trips is not only used for driving deadheads. As electric vehicles generally have relatively small ranges when compared to their combustion based counterparts, this downtime is also used in order to recharge the vehicle. A variety of different recharging methods are currently in use throughout the industry, however in this work we will exclusively consider conductive charging; specifically, we consider plug-in charging (which is the same method used for commercial EVs) and pantograph charging. These two techniques share many similarities in their behavior, with the largest difference being the physical connector that is used to connect the power source to the battery. For a detailed overview of available recharging methods and corresponding behavior, we refer the reader to a review by \citet{Zhou2024}.

For conductive charging, recharging stations are often available at the depot, with additional stations possibly being present at the starting or ending locations of trips (also called the terminal trip stops). Charging stations in the middle of a trip are generally not present. Recharging a vehicle at one of these locations follows a charging curve; this curve defines the charging rate when a vehicle is at a certain state of charge. For the most commonly used battery chemistry, lithium-ion, this curve is generally split into two phases: from 0 to around 80\%, in which the charging rate is roughly constant, and from 80 to 100\%, where it slows significantly. These phases often take similar amounts of time, and depending on the infrastructure available a full charge may take on the order of hours. Fast-charging infrastructure has become more prevalent in recent years with charging speeds of up to $\sim$350km/hour, however this still results in significant time spent charging during a vehicle task.

The resulting vehicle downtime therefore necessitates careful planning of crew schedules. Crew members must follow local labor regulations, which often consist of maximum driving times throughout the day and required breaks. Combining these breaks with charging time for the vehicle can therefore be beneficial. Additionally, crew members can use multiple vehicles throughout the day, further reducing crew downtime if a certain vehicle is required to wait for a trip.

Vehicle handovers are often only possible at a predetermined set of locations. These relief location have waiting areas or break rooms connected to them, providing a place for a crew member to stay before entering their next vehicle. As not all locations are suitable for handovers, a sequence of trips and deadheads which does not visit a relief point must be driven by a single driver. Such a sequence which starts and ends at a relief location is called a block, and represents the smallest unit of work that can be performed by a driver. Both vehicle and driver schedules can be seen as a collection of blocks, with possible idle time and break times to connect the blocks.

Lastly, drivers may sometimes travel between different locations by means other than driving a vehicle. An example of this may be when locations are geographically close, allowing a driver to walk between them if required. This provides additional flexibility in handover procedures, and allows for a central break location to be present when multiple stops are close to one another.

Overall, the goal is to find a set of feasible vehicle and crew schedules that minimizes costs. In this, vehicle schedules must ensure that each trip is driven while minimizing deadhead and charging costs. These schedules must also ensure that enough charging takes place such that the vehicle state of charge is within acceptable levels. Crew schedules, on the other hand, must ensure that each driven block is covered. While doing so, local labor regulations must be taken into account.

\begin{table}[h]
  \centering
  \begin{tabular}{ll}
    \toprule
    \multicolumn{1}{l}{\textbf{Notation}} & \multicolumn{1}{l}{\textbf{Definition}}               \\
    \cmidrule(lr){1-1}\cmidrule(lr){2-2}
    \multicolumn{2}{l}{\textit{Given}} \\
    $\mathcal{D}$ & The depot \\ 
    $t \in \mathcal{T}$ & Set of all trips, $t$ a single trip \\
    $dh \in \mathcal{DH}$ & Set of all feasible deadheads, $dh$ a single deadhead \\ 
    $b \in B$ & Set of all blocks, $b$ a single block \\
    $v \in V$ & Set of all vehicle tasks, $v$ single vehicle task \\
    $c \in C$ & Set of all crew tasks, $c$ single crew task \\
    $d_\chi$ & Amount of charge used driving a trip/deadhead \\
    $u_{dh,\sigma}$ & Max. charge gained in deadhead $dh$ with starting SoC $\sigma$ \\
    $\sigma_{start}$ & SoC at beginning of block/task \\
    $\sigma_{min}$ & Minimum SoC of vehicle \\
    $\sigma_{max}$ & Maximum SoC of vehicle \\ 
    $k_{i,t}$ & Binary constant, vehicle task $v_i$ covers trip $t$ \\ 
    $l_{i,b}$ & Binary constant, vehicle task $v_i$ uses block $b$ \\ 
    $m_{j,b}$ & Binary constant, crew task $c_j$ uses block $b$ \\ 
    \addlinespace[0.6em]
    \multicolumn{2}{l}{\textit{Decision variables}} \\
    $x_{i} \in \{ 0, 1 \}$ & Usage of vehicle task $v_i$  \\ 
    $y_{j} \in \{ 0, 1 \}$ & Usage of crew task $c_j$ \\ 
    \addlinespace[0.6em]
    \multicolumn{2}{l}{\textit{Helper functions}} \\
    $b(v) \subseteq B$ & Set of blocks covered by $v$ \\ 
    $km(v)$ & Kilometers traveled during $v$ \\ 
    $chg(v)$ & Charge gained during $v$ \\ 
    $chl(v)$ & Charge lost during $v$ \\ 
    $br(c)$ & Total break time in $c$ \\ 
    $dur(\chi)$ & Total duration of $\chi$ \\ 
    $cost(\chi) \in \mathbb{R}^+$ & Cost of task $\chi$ \\ 
    \addlinespace[0.2em]
    \bottomrule
  \end{tabular}
  \caption{Notation used for formal problem description, where $\chi$ is used as a placeholder when multiple argument types can be applied.}
  \label{tab:notation}
\end{table}

\subsection{Formal definition}
We will now formally define the E-VCSP. In this, we will consider the variant with a single depot, single vehicle type and charging stations without capacity limitations. Additionally, partial and non-linear charging are considered. A summary of the notation used has been provided in Table \ref{tab:notation}.

Given timetables for each of the considered lines, let $\mathcal{T}$ be the set of trips that they define. Additionally, let $\mathcal{DH}$ be the set of deadheads that represent either a travel to or from the depot $\mathcal{D}$, or a travel between trips in $\mathcal{T}$. In this, only deadheads are present which can feasibly be driven; that is, the travel time of the deadhead must be less than the time between its departure and the start of a target trip. Note that deadheads including a depot as either the origin or target are always considered to be time-feasible.

Next, let us define our battery behavior. For each trip $t$ and deadhead $dh$, let $d_t$ and $d_{dh}$ respectively indicate the amount of charge that is used during its travel. Additionally, let $u_{dh,\sigma}$ indicate the maximum amount of charge that can be gained during a deadhead if a vehicle starts the deadhead at a State of Charge (SoC) of $\sigma$; in this, let charging at any possible given charging location be considered. Depending on available charging infrastructure, this may imply that a deadhead is not driven directly, instead making a time-feasible detour via a charging location which provides the fastest charging capabilities. SoC gained at a location depends on given (possibly non-linear) charging curves for the vehicle, allowing for different amounts of charge to be achieved in the same time span depending on the initial SoC. Note that when discrete SoC values are used, all values of $u_{dh,\sigma}$ can be precomputed.

Lastly, let us define relief points and crew bases. A subset of the terminal locations of trips may serve as relief points; these locations are the only ones at which a handover or break for a crew member can take place. A subset of the relief points is a crew base; these locations are the only ones at which a crew member may start or end their duty.

Let a \textit{block} be a sequence of trips and/or deadheads that must be driven by a single driver; that is, no relief points are visited during the sequence except at its start and end. Let an individual trip or deadhead in this sequence be called an \textit{element}. In order for a block to be feasible, it must match the following requirements: 
\begin{itemize}
  \item A block consists of at least one trip or deadhead.
  \item Only the starting location of the first element and the ending location of the last element in the block are relief points.
  \item The element directly following a trip must be a deadhead and vice versa. 
  \item For each deadhead $dh$ driven, the vehicle may charge with some value $\alpha \cdot u_{dh,\sigma}$, where $0 \leq \alpha \leq 1$ represents a fraction of the maximal charge gained. 
  \item Given an SoC $\sigma_{start}$ at the beginning of the block, the SoC of the vehicle should remain within the range $[ \sigma_{min}, \sigma_{max} ]$ throughout the block.
\end{itemize}
Let the set of all feasible blocks be $B$. A schedule for a vehicle throughout the day can now be constructed as a sequence of blocks. Let us refer to such a schedule as a \textit{vehicle task}. In order for a vehicle task to be feasible, the following conditions need to be met:
\begin{itemize}
  \item The task must start and end at the depot. 
  \item A vehicle task consists of one or more blocks. 
  \item Sequential pairs of blocks must be compatible; that is, for each sequential pair of blocks $b$ and $b'$, $b$ must either end in a deadhead whose target is the starting trip of $b'$, or $b$ must end in a trip which is the origin of the starting deadhead of $b'$.
  \item Given an SoC $\sigma_{start}$ at the beginning of the task, the SoC of the vehicle should always remain within $[ \sigma_{min}, \sigma_{max} ]$ during all blocks.
\end{itemize}
Let the set of all feasible vehicle tasks be $V$. For a vehicle task $v$, let us define its cost as being $cost(v) = \gamma_1 + \gamma_2 \cdot km(v) + \gamma_3 \cdot chg(v) + \gamma_4 \cdot chl(v)$, where $km(v)$, $chg(v)$ and $chl(v)$ represent kilometers traveled, charge gained and charge lost during $v$ respectively and $\gamma$ is a vector of cost scalars.

Next, let us consider crew schedules. As with vehicle tasks, crew schedule primarily consist of blocks that need to be driven throughout the day. Crew members are however bound by labor regulations; in most cases, this means that we will need to include break times and a maximum overall duration in the schedule. Additionally, idle time at handover locations may be required in order to wait for an arriving vehicle, and crew members may travel to different locations through means other than driving a vehicle. A \textit{crew task} therefore consists of a sequence of blocks, idle/break times, and travels between locations without driving. As with vehicle tasks, we will refer to individual parts of this sequence as elements. Exact feasibility constraints for crew tasks will differ per country and region based on local labor regulations, however in general the following conditions must always be met: 
\begin{itemize}
  \item The crew task consists of one or more blocks and zero or more travels and break/idle times.
  \item Sequential pairs of elements must be compatible. That is, for each sequential pair of elements $e$ and $e'$, the location at the end of $e$ must be the same as the start of $e'$. Additionally, the ending time of $e$ must be equal to the starting time of $e'$. 
  \item The crew task must start and end at the depot.
\end{itemize}
In addition to these universal constraints, we also incorporate a simplified version of the Dutch labor regulations into our feasibility: 
\begin{itemize}
  \item The crew task may have a duty length of at most 9 hours 
  \item A crew member may drive for at most 4 hours without a break.
  \item For tasks between 4 and 5.5 hours, at least one break of 15 minutes must be included. 
  \item For tasks longer than 5.5 hours, at least 40 minutes of break time must be included. All breaks must be longer than 15 minutes, and one break must be at least 20 minutes.
  \item For tasks starting after 15:00, at least one break between 16:30 and 20:30 of at least 20 minutes must be included. 
\end{itemize}
Let the set of all crew tasks that match these conditions be $C$. For a crew task $c$ let us define its cost as being $cost(c) = \beta_1 + \beta_2 \cdot br(c) + \beta_3 \cdot dur(c)$, where $br(c)$ and $dur(c)$ represent the total break time in and duration of $c$ respectively and $\beta$ is a vector of cost scalars. \\

Using the definitions of the blocks $B$, vehicle tasks $V$ and crew tasks $C$, we can formulate a linear program to describe our integrated problem. In this, let $x_i$ and $y_j$ be binary variables indicating the usage of $v_i \in V$ and $c_j \in C$ respectively. Additionally, let $k_{i,t} \in \{ 0, 1 \}$ indicate that $v_i$ covers trip $t \in \mathcal{T}$, let $l_{i,b}$ indicate that $v_i$ contains block $b \in B$, and let with $m_{j,b} \in \{ 0, 1 \} $ indicate that $c_j$ includes block $b \in B$. In this, $k_{i,t}$, $l_{i,b}$ and $m_{j,b}$ are all parameters. Our cost minimization can then be formulated as follows:
\begin{align}
\min \quad
& \sum_{1 \leq i \leq |V|} x_{i} \cdot cost(v_i) + \sum_{1 \leq j \leq |C|} y_{j} \cdot cost(c_j)  
\end{align}
Subject to:
\begin{align}
\sum_{1 \leq i \leq |V|} x_{i}k_{i,t} &= 1 && \forall t = 1,\:\dots,\:|T| \label{form:all-trips-covered} \\
\sum_{1 \leq i \leq |V|}x_i l_{i,b}  &= \sum_{1 \leq j \leq |C|}y_j m_{j,b}  && \forall b \in B \label{form:all-blocks-covered} \\
x_{i} &\in \{ 0, 1 \} && \forall i = 1,\:\dots,\:|V| \\
y_{j} &\in \{ 0, 1 \} && \forall j = 1,\:\dots,\:|C|
\end{align}
Here, constraint (\ref{form:all-trips-covered}) ensures that all trips are covered by a vehicle, and constraint (\ref{form:all-blocks-covered}) ensures that for each block, the number of assigned drivers is equal to the number of assigned vehicles. This integrated formulation minimizes the costs for both vehicle and crew tasks simultaneously. For comparison, when the E-VSP and CSP are solved sequentially, the formulation becomes the following. First, vehicle tasks are selected such that their costs are minimized:
\begin{align}
\min \quad
& \sum_{1 \leq i \leq |V|} x_{i} \cdot cost(v_i)
\end{align}
Subject to:
\begin{align}
\sum_{1 \leq i \leq |V|} x_{i}k_{i,t} &= 1 && \forall t = 1,\:\dots,\:|\mathcal{T}| \label{form:all-trips-covered-seq} \\
x_{i} &\in \{ 0, 1 \} && \forall i = 1,\:\dots,\:|V|
\end{align}
Here, constraint (\ref{form:all-trips-covered-seq}) once again ensures that all trips are covered by a vehicle task. Once the vehicle tasks are selected, we can use them to derive the set of blocks that are driven. Let $b(v) \subseteq B$ represent the set of blocks driven by a vehicle task $v$. The overall set of blocks driven is then $B' = \bigcup_{v_i \in V, x_i = 1}b(v_i)$, and let the crew tasks for which all used blocks appear in $B'$ be $C'$. Note that in our construction of $B'$, each block is present at most once due to the fact that all trips are covered exactly once in our E-VSP formulation. \\
Using this, our formulation for crew tasks becomes the following:
\begin{align}
\min \quad
& \sum_{1 \leq j \leq |C'|} y_{j} \cdot cost(c_j)  
\end{align}
Subject to:
\begin{align}
\sum_{1 \leq j \leq |C'|} y_j m_{j,b} &= 1  && \forall b \in B' \label{form:all-blocks-covered-seq} \\
y_{j} &\in \{ 0, 1 \} && \forall j = 1,\:\dots,\:|C'|
\end{align}
Here, we now consider a greatly reduced number of available blocks and crew tasks, resulting in a formulation with significantly less rows and columns than the equivalent integrated formulation.

\section{Related work}
In this section, we discuss work related to our research into the E-VCSP. A summary of how batteries and charging behavior are modeled in the discussed works has been included in Table \ref{tab:eVCSP-lit}.

\afterpage{
  \clearpage
  \begin{landscape}
  \null
  \vfill
  \begin{table}[h!]
    \centering
    \begin{tabular}{llllllll}
      \toprule
                                        & Model   & ToU & SoC & Nonlinear Ch. & Partial Ch. & Ch. Location & Degradation \\
      \cmidrule(lr){2-8}
      \citet{Li2014}               & E-VSP   & No  & D   & No            & No          & D            & No          \\
      \Citet{vanKootenNiekerk2017} & E-VSP   & Yes & C/D & Yes           & Yes         & D/T          & Yes         \\
      \citet{Olsen2020}            & E-VSP   & No  & C   & Yes           & Yes         & D/T          & No          \\
      \citet{Zhang2021}            & E-VSP   & No  & C/D & Yes           & Yes         & D            & Yes         \\
      \citet{Parmentier2023}       & E-VSP   & No  & C   & Yes           & Yes         & D/T          & No          \\
      % \citet{Pulyassary2024}       & E-VSP   & No  & C/D & Yes           & Yes         & T            & No          \\
      \Citet{deVos2024}            & E-VSP   & No  & D   & Yes           & Yes         & D/T          & No          \\
      \addlinespace[0.4em]
      \citet{Perumal2021}          & E-VCSP  & No  & C   & No            & No          & D            & No          \\
      \citet{Wang2022}             & E-VCSP  & Yes & C   & No            & Yes         & D            & No          \\
      \citet{Sistig2023}           & E-VCSP  & No  & C   & No            & Yes         & D/T          & No          \\
      \citet{Shen2023}             & E-VCSP  & No  & C   & No            & Yes          & D/T          & No          \\
      \citet{Cong2024}             & E-VCSP  & Yes & C   & No            & Yes         & D            & No          \\
      \addlinespace[0.4em]
      \citet{Ham2021}              & E-VRPTW & Yes & C   & No            & Yes         & D            & No          \\
      \citet{Stadnichuk2024}       & E-TVSP  & No  & C   & No            & Yes         & D/T          & No          \\
      \bottomrule
    \end{tabular}
    \caption{A brief overview of battery modeling in E-VCSP related literature. SoC modeled as (D)iscrete or (C)ontinuous variable, Charge locations at (D)epot or (T)erminal trip stops, Degradation of battery in cost function}
    \label{tab:eVCSP-lit}
  \end{table}
  \vfill
  \end{landscape}
  \clearpage
}

\subsection{(E-)VSP}
Before considering previous work on the E-VSP, we first cover the most basic form of vehicle scheduling: that which only considers a single depot, single vehicle type and unlimited vehicle ranges. This problem, often referred to as the Single Depot Vehicle Scheduling Problem (SDVSP), forms the underlying basis of both the multi-depot and electric vehicle extensions that we consider later. We therefore give a brief summary of two common models and solution methods used for the SDVSP, thereby having a baseline to which we can compare extensions. For a more comprehensive overview on different models used for the SDVSP and Multi-Depot VSP (MDVSP), we refer the reader to a review by \citet{Bunte2009}.

In order to find a solution for the SDVSP, the problem can be transformed into one of finding a min-cost flow in a graph. The graph can be constructed as follows: For all trips $\mathcal{T}$, add a pair of nodes representing the start and end of the trip respectively. Add an arc from the start of each trip to its end with capacity 1, lower bound 1 and cost equal to that of driving the trip. Next, connect the end of each trip $t$ to the start of each trip $t'$ for which the deadhead between the two is feasible; that is, there is enough time to drive the deadhead from $t$ to $t'$ before the scheduled starting time of $t'$. Let each of these arcs have a capacity 1 and a cost equal to that of driving the deadhead from $t$ to $t'$. Lastly, let us introduce a pair of nodes representing the depot at the start and end of the day respectively. Connect the depot start node to the start of each trip, and do the same for the depot end node and the end of each trip. For each of these arcs, let the capacity once more be 1 and let the cost be equal to that of driving of the deadhead between the depot and trip. Fixed vehicle costs can be represented by adding costs to the arcs leaving the depot.

We can now find the min-cost flow in this graph; in this, let the depot start node be the source, and let the depot end node be the sink. Due to our construction, all trips must be covered by exactly 1 flow, resulting in flow paths which we can directly use as vehicle tasks due to the assumption that our vehicles have infinite range. It is therefore also shown that the SDVSP can be solved in polynomial time, as polynomial time min-cost flow algorithms exist and the graph elements are of size $|V| = O(|\mathcal{T}|)$ and $|A| = O(|\mathcal{T}|^2)$.

Alternatively, the problem can be transformed into one of finding a minimum weight matching within a bipartite graph. Here, nodes are created for each trip on either side of the of the graph, where the left side of the graph represents origins and the right side represents destinations. Additionally, for each available vehicle, a depot node is added on either side of the graph. An edge between a origin-destination pair is added when the deadhead between them is feasible, and the cost of driving the deadhead is used as its weight. A minimum weight matching in this graph can then be used to construct vehicle tasks by following origin-destination pairs from a depot node on the left side to a depot node on the right side.

Two common extensions to the SDVSP make it NP-Hard: the inclusion of multiple vehicle types, as well as the use of multiple depots under the assumption that vehicles must return to their depot of origin. Both of these extensions are also discussed in \citet{Bunte2009}. The modification to the SDVSP flow network is the same in either case: an additional source/sink pair can be added for each new depot or vehicle type, and connected to the trips in the same way as the original depot. The problem then turns into into an integral multi-commodity flow, which has been shown to be NP-Hard by \citet{Even1975}.

The introduction of any resource constraints within the VSP has also been shown to be NP-Hard (see \citet{Bodin1983}). The E-VSP specifically deals with constraints on the driving range of vehicles, thereby making it closely related to the vehicle scheduling problem with route time constraints (VSP-RTC) as described by \citet{Haghani2002}. The key difference between these two problems is that the E-VSP allows for (partial) recharging of a vehicle throughout the operating period, whereas the VSP-RTC assumes a fixed maximum travel time for the vehicle within the given period. The E-VSP has been shown to be NP-Hard by \citet{Sassi2014}. \\

\citet{Li2014} is one of the first to consider a solution method for the E-VSP. They consider a single-depot case with a single vehicle type, in which the assumption is made that full recharging (or battery swaps) can be performed in a fixed 5-minute time window. The model is based on the SDVSP flow network, with the inclusion of driving time constraints for the vehicle tasks. Additionally, time-discretized nodes are added to represent capacitated battery charging/swap stations. Connections are made from the trip nodes to the charging nodes when a travel between the two is feasible, allowing for a vehicle to perform one or more charging action during its task. By using discretized charging nodes, charging station capacity is enforced with the use of flow constraints. For smaller instances, the model can be solved to optimality using column generation and branch-and-price (B\&P). For larger instances, an alternate approach using truncated column generation followed by a local search to find a local optimum is used instead. The proposed methods are tested on trips in the San Francisco Bay Area, with a maximum instance size of 242 trips. These tests resulted in optimality gaps of $<5\%$ for buses able to drive 150km, and between 7-15\%  for a range of 120km depending on the instance.

\Citet{vanKootenNiekerk2017} introduce two models which aim to solve the single depot E-VSP while taking into account time dependent energy prices (ToU pricing), nonlinear charging times and battery degradation due to depth of discharge. The first model extends the traditional SDVSP flow model by discretizing the depot nodes over time, and adding a continuous variable to each trip node representing the vehicle SoC at its start. This only allows for the formulation that uses linear charging curves, and does not incorporate degradation or ToU. The second model additionally discretizes both depot and trip nodes for individual starting SoC values. Charge-feasibility of deadheads can then be considered during graph construction, only adding arcs between pairs of nodes when the SoC difference between them can be achieved through driving and charging during the deadhead. This also allows for easy integration of nonlinear charging curves, degradation, and ToU pricing, as the SoC of the vehicle at each discretized trip node is known at construction time. The second model is solved using CG and Lagrangean relaxation. Tests are performed using data provided by Belgian bus company De Lijn in the city Leuven, using a total of 543 trips. They show that the second model can be solved in a considerably shorter time frame for large instances with similar results to the first.

\citet{Olsen2020} consider a multi-depot E-VSP, in which they model the nonlinear phase of charging as an exponential function. In order to solve, they implement a greedy heuristic to construct vehicle tasks. Their primary focus is comparing (piecewise) linear approximations for the second phase of charging with an exponential function based approximation. They conclude that SoC and required charging times are more comparable to real life behavior when using the exponential function.

\citet{Zhang2021} consider an E-VSP variant with a single depot, capacitated charging infrastructure, non-linear charging behavior and battery degradation. They model this by combining elements from \citet{Li2014} and \citet{vanKootenNiekerk2017}, resulting in a network which has time and charge discretized depot, trip and charging station nodes; arcs between these are again only present if the deadhead is both time and charge-feasible. Solutions are found using a combination of CG and B\&P, and tests are performed on both randomly generated instances as well as 6 not yet electrified lines with up to 160 and 197 trips respectively.

\citet{Parmentier2023} consider a scalable approach to the E-VSP with non-linear charging. They introduce the concept of nondominated charging arcs, which are represented as multiple deadhead arcs between a pair of nodes within the traditional SDVSP network. Their use allows for a manageable amount of charging possibilities to be considered between trips when multiple charging points are available, as an arc is only included if there is not another arc available with higher resulting charge and lower cost. In order to solve, a combination of CG and B\&P techniques are used. Testing is done on the \textit{large} instances introduced by \citet{Wen2016} which included up to 8 depots, 16 charging stations and 500 trips. Here, they are able to find solutions that only have an 0.06\% optimality gap.

\Citet{deVos2024} consider the E-VSP with partial recharges and capacitated charging stations. Their model applies a similar discretization as the ones found in the work of \citet{vanKootenNiekerk2017} and \citet{Zhang2021}. As with those models, power used during trips and deadhead arcs is rounded up to the nearest discrete value; this results in an underestimation of the actual SoC of the vehicle during its task, however ensures solutions that can be feasibly driven. This pessimistically rounded graph, which De Vos et al. refer to as the primal network, is accompanied by a dual network; in this graph, power used is rounded down, resulting in more deadheads becoming charge-feasible. The problem is then solved by applying CG with two separate approaches: branch-and-price and a diving heuristic. In this, the dual network is used in order to generate dual bounds that match those found in a non-discretized model, following ideas presented by \citet{Boland2017}. Testing is performed on a bus concession south of Amsterdam with 816 trips, with subsets being used as smaller instances. Optimality gaps of 1.5-2.7\% are achieved across instances. They additionally note that the framework as provided can easily be extended for nonlinear charging functions and depth-of-discharge battery degradation. 

\subsection{CSP}
Given a solution to the (E-)VSP, the corresponding CSP is most often solved as a set partitioning (or set covering) problem. Here, the tasks described by the sequences of trips generated during vehicle scheduling must be covered by the individual schedules of crew members. This problem has been shown to be NP-Hard in general by \citet{Fischetti1989}.

Research into this subject is primarily done in the context of airline crew planning; crew costs in this field are generally even higher than those found in the more general public transport sector, as shown in \citet{Barnhart2003}. Additionally, strong labor unions and restrictive labor legislation due to safety concerns cause a large number of constraints to be applied to crew schedules, resulting in a non-trivial problem to solve.

Results achieved in the aviation space quite easily generalize to other sectors, and we therefore refer the reader to a review by \citet{Wen2021} for an overview of the state of the art. 

\subsection{(E-)VCSP}
The VCSP is a widely studied problem. Following the call for integrated methods by \citet{Bodin1983} and others in the 1980s, a large number of different methods has been applied to integrate the VSP and CSP. We refer the reader to a recent review by \citet{Ge2024} for a general overview of work done in the field in the past years.

One work that we will individually highlight is that of \citet{Huisman2005}, due to its use of Lagrangean relaxation to connect the VSP and CSP . For readers unfamiliar with the technique, we recommend an introduction by \citet{Beasley1993}. Huisman et al. consider the multi-depot variant, and use a combination of CG and Lagrangean relaxation to solve both the MDVSP as well as the connection with the CSP. Of note is their assumption that crew members from each individual depot are only allowed to work on trips connected to said depot, allowing for individual depot CSPs to be solved as a subproblem. They test on instances in the Randstad metro area in the Netherlands with a maximum of 653 trips and 4 depots.


As for the electric counterpart of the VCSP, at time of writing we are aware of only five other works that discuss the integrated variant.

\citet{Perumal2021} were the first to offer a solution to the E-VCSP. They consider an instance of the problem in which only full recharges at the depot with a fixed duration of 120 minutes are possible. Solutions are found using ALNS, incorporating a B\&P heuristic for constructing tasks which has been previously used to solve the VCSP, MDVSP and E-VSP by \citet{Haase1996}, \citet{Pepin2009} and
\citet{ vanKootenNiekerk2017} respectively. The authors tested using real life data from lines in Denmark and Sweden with a
maximum instance size of 1109 trips and multiple depots, and report an improvement of $1.17-4.37\%$
across different instances when compared to a sequential approach.

\citet{Wang2022} introduce a two layered model which allows for a mix of traditional combustion and electric buses using Particle Swarm Optimization and an $\epsilon$-constraint based mechanism. The model incorporates partial depot charging, as well as measures to ensure that crew is primarily assigned to the same vehicle throughout the day. A circular bus route with a single depot in Changchun, China with 68 daily trips is used as a basis for testing, with a focus on electric versus diesel usage and driver satisfaction.

\citet{Sistig2023} also offer an ALNS based approach, which aims to incorporate additional flexibility in vehicle scheduling when compared to the work presented by \citet{Perumal2021} by including partial recharges, opportunistic charging at terminal stops of trips and non-fixed ranges for the vehicles. A set of 3-step ALNS neighborhoods are used in order to solve, where each neighborhood follows the structure of first doing a number of improvement iterations to the E-VSP solution, deriving the corresponding CSP problem, before finally doing a number of improvement iterations on the CSP solution. The E-VSP and CSP iterations themselves also follow a ALNS structure, resulting in nested ALNS iterations being performed. The upper level 3-step neighborhoods only differ in the amount of iterations done in each of the two nested ALNS. This approach is tested using an instance of a city route in Germany, with a single depot and a total of 282 trips. Different scenarios based on possible crew break and relief locations were considered in order to compare diesel and electric TCO. Additionally, sensitivity analysis of the TCO was done for parameters such as costs for electricity and drivers.

\citet{Shen2023} provide a minimum-cost flow based framework for the E-VCSP. They consider the single depot variant, in which partial recharges are possible. Solutions are found by first generating a subset of all feasible blocks and corresponding crew tasks using a matching-based heuristic approach. Afterwards, a MIP formulation is used in order select crew tasks and create corresponding vehicle tasks that cover all trips. A city line in China with 270 daily trips and a single depot is used for testing, resulting in cost savings of up to 8.7\% when compared to a sequential approach.

\citet{Cong2024} take a hybrid MIP and SAA based approach to optimize a mixed fleet of combustion and electric vehicles with ToU electricity pricing. In each SAA iteration, a collection of new E-VSP trip assignments are created using neighborhood operations, after which two MIP models are sequentially employed to solve for charging and crew schedules. The methods are tested on a collection of 3 bus routes originating from the same depot in Changchun City, China with a total of 520 trips across all routes. When compared to the sequential approach, the integrated vehicle schedule was able to reduce costs by 0.8\%. 

\subsection{Other related fields}
The VSP is closely related to the vehicle routing problem (VRP); in this problem, the aim is to find minimum cost routes for vehicles originating from a depot that need to pass multiple stops, most commonly for pickup or delivery with capacity constraints. The extension of the E-VRP which includes arrival time windows (E-VRPTW) is most closely related to the E-VSP, as the use of 0-width windows allows us to define the same precedence constraints as those naturally defined by trips in the VSP.

An example of work done on the E-VRPTW is that of \citet{Ham2021}. They consider a single depot case in which they model ToU pricing and partial recharges during delivery routes. In order to model costs, a lexicographical minimization is done over the number of vehicles used, total distance traveled and energy recharged. In order to solve, a hybrid MIP and CP algorithm is used in which CP is used to model ToU related variables, and MIP is used to model the rest of the constraints. \\

Research has also been done into integrating the E-VSP with the step before it in the planning sequence: timetable planning. This problem, the E-TVSP, has recently been studied in the work of \citet{Stadnichuk2024}. They allowed results of the E-VSP to introduce optimality cuts into the MIP used for creating timetable plans, thereby reducing overall cost. This is achieved by transforming the E-VSP problem into one of bin packing with conflicts, after which three different heuristic methods are applied and compared. They additionally prove that the bounds of the used heuristics are tight for their given instances. 

\subsection{Research gap}
As can be seen in Table \ref{tab:eVCSP-lit}, research into the E-VSP has successfully incorporated many battery characteristics which can be of importance when determining overall operational costs and feasibility. The E-VCSP on the other hand has seen less progress: both non-linear charging and battery degradation have not yet been considered in any work at the time of writing. Additionally, no documented attempt at using a discretized SoC model for the E-VCSP has been made. In this work, we therefore aim to address two of these points: the use of a discretized model, as well as incorporating non-linear charging curves. 

\section{Methodology}
In this section, we will discuss our methodology for solving the E-VCSP. In order to do so, we will first discuss our ways of solving the E-VSP and CSP sequentially; afterwards, we discuss how parts of these techniques can be reused in order to solve the integrated problem.

For both the E-VSP and CSP a (truncated) column generation approach is implemented based on a set cover ILP. In order to make this approach computationally feasible, a combination of pre- and postprocessing is applied on the underlying networks and solutions. As large parts of the solution methods for these two problems are shared, a detailed overview of the any shared technique will only be provided in the E-VSP section. Where applicable, the CSP solution method will refer back to this. 

\subsection{E-VSP}
Our approach for solving the E-VSP can roughly be broken up into 3 steps: 
\begin{itemize}
  \item \textit{Preprocessing}, in which our initial data set is transformed into a graph which is simplified using heuristics.
  \item \textit{Solving}, in which truncated column generation using various sources is applied in order to solve the problem of finding a set of vehicle tasks which cover all trips. 
  \item \textit{Postprocessing}, in which duplicate trips are transformed into deadheads in order to reduce the total amount of driving manhours required. 
\end{itemize} 
\subsubsection{Preprocessing}
Given are a set of trips $\mathcal{T}$, a set of locations $\mathcal{L}$, and a set of deadheads $\mathcal{DH}$ connecting pairs of locations. Of these locations, exactly one is the depot $\mathcal{D} \in \mathcal{L}$. Additionally, a subset of the locations $\mathcal{L}_c \subseteq \mathcal{L}$ has charging equipment installed.

Inspired by \citet{vanKootenNiekerk2017}, let us transform this data into a graph $G_{vt} = (V_{vt}, A_{vt})$ in which paths represent individual vehicle tasks. Let $V_{vt} = \{ d_{s}, d_{e} \} \cup \mathcal{T}$, where $d_{s}$ and $d_{e}$ represent the depot at the beginning and end of the day respectively. For each node $t \in \mathcal{T}$, add an arc to $A_{vt}$ from $d_{s}$ to $t$ and from $t$ to $d_{e}$ if there is a deadhead in $\mathcal{DH}$ connecting the depot to the starting and ending location of trip $t$ respectively. Next, add an arc to $A_{vt}$ between each pair of trips $t, t'$ if there is a deadhead in $dh \in \mathcal{DH}$ that connects the ending location of $t$ to the starting location of $t'$ and the duration of $dh$ does not exceed the time between the end of $t$ and the start of $t'$. This graph corresponds to the one found in a standard VSP formulation; we will now modify it in order to be more computationally efficient and include additional charging capabilities.

Let us begin by reducing the overall size of the graph. Oftentimes a timetable will include sequences of trips which are meant to be driven sequentially. This most commonly occurs when a line is driven in two directions; a trip in one direction may be followed by a trip in the other direction, implying that a vehicle turns around and drives both directions sequentially. This can be used to substantially reduce the amount of arcs in the graph by forcing the use of short turnaround arcs between trips if they are available. 

More formally: for each trip node $t$, let $A_t \subset A_{vt}$ be the set of outgoing arcs from $t$ to other trip nodes. Let $A_{t,st} \subseteq A_t$ be the collection of arcs for which the time between $t$ and the target trip $t'$ is less than some threshold $\delta$ and for which the end location of $t$ is equal to the starting location of $t'$. If $|A_{t,st}| \geq 1$, replace $A_t$ with $A_{t,st}$. Otherwise, leave the original set of outgoing arcs the same. A $\delta$ in the order of $\sim 0.5$ hours was found to be reasonable during initial testing.

Once this initial reduction of arcs between trips has been performed, we will extend the remaining arcs in order to properly accommodate charging. Let us first assume that charging never occurs when driving to or from the depot; all charging operations will therefore take place in the time between pairs of trips. Given the current state of our graph $G$, this implies that charging may only take place if a trip starts or ends at a charging location, and a connecting trip allows for idle time which can be used for charging.

In order to increase the amount of charging options, let us consider the existing arcs in $A_{vt}$ between pairs of trips $t$ and $t'$; let us refer to these arcs as \textit{base arcs}. For each charging location $l_c \in \mathcal{L}_c$ for which $l_c$ is not the ending location of $t$ or the starting location of $t'$, we will attempt to add an additional charging arc. This arc will model the possibility of performing a detour between $t$ and $t'$ to $l_c$ in order to charge there; let us refer to this as a \textit{detour arc}. Let us add this detour arc if there exists a deadhead $dh \in \mathcal{DH}$ from the ending location of $t$ to $l_c$, a deadhead $dh' \in \mathcal{DH}$ from $l_c$ to the start of $t'$, and the combined duration of $dh$ and $dh'$ allows for enough time at $l_c$ to meet the minimum charging time.

For all detour arcs, we assume that the time at the charging location is maximized. This means that we we always perform the deadhead towards the charging location at the earliest possible time, and the deadhead to the next trip at the latest possible time. For base arcs, let us also maximize time at a charging location; if the start and end location of a base arc are both non-charging or both charging locations, time at the start location is maximized as a tiebreaker. Knowing this, an arc can be uniquely defined as including: 
\begin{itemize}
  \item Overall driving costs for the movements (deadhead(s) / target trip).
  \item SoC usage for each individual movement.
  \item Time spent at a predetermined charging location, if one is visited. If the vehicle is not fully charged, this time will always be used. \todo{beter plaatsen}
\end{itemize}

Any feasible vehicle task can now be described as a path in $G$ from $d_s$ to $d_e$. The costs for a certain task can be defined as the sum of all used arcs driving costs + a KWh cost for the sum of all SoC usage. In this, we assume that the cost of energy remains constant throughout the day and at different charging locations, and that vehicles will be fully recharged overnight for the same KWh cost as during the day. Additionally note that not every path from  $d_s$ to $d_e$ describes a feasible vehicle task; depending on the arcs used, the amount of time that a vehicle has to charge may not be sufficient in order to keep it within SoC bounds during the task. 

\todo{Misschien gewoon een voorbeeld? Wat trips, deadheads en charging locaties in tabel vorm, resulterende graaf met evt dubbele arcs voor detour arcs}

\subsubsection{Solving}
In order to now solve the E-VSP, a collection of feasible vehicle tasks that cover the trips $\mathcal{T}$ must be found. In order to do this, we will use a (truncated) column generation approach; this has previously been shown to be an effective solution method for the E-VSP by works such as \citet{vanKootenNiekerk2017}, \citet{Zhang2021} and \citet{deVos2024}. The ILP used to model this is a set covering ILP; in this let $VT$ represent the available set of vehicle tasks, let $x_{i}$ be a binary decision variable indicating the use of vehicle task $v_i \in VT$, and let $k_{i,t}$ be a parameter defining whether vehicle task $v_i$ includes the trip $t \in \mathcal{T}$. 
\begin{align}
\min \quad
& \sum_{1 \leq i \leq |VT|} x_{i} \cdot cost(v_i)
\end{align}
Subject to:
\begin{align}
\sum_{1 \leq i \leq |VT|} x_{i}k_{i,t} &\geq 1 && \forall t = 1,\:\dots,\:|\mathcal{T}| \label{form:vehicle-cover}\\
x_{i} &\in \{ 0, 1 \} && \forall i = 1,\:\dots,\:|VT| \label{form:x-integer}
\end{align}
This problem can be relaxed to an LP by replacing constraint (\ref{form:x-integer}) with the following:
\begin{align}
x_{i} &\geq 0 && \forall i = 1,\:\dots,\:|VT|
\end{align}
In this relaxed problem, let the dual cost of constraint (\ref{form:vehicle-cover}) be $\pi_t$ with $t \in \mathcal{T}$. 

The LP relaxation can initially be solved using dummy columns each containing exactly 1 trip. Afterwards, new columns can be are generated, guided by the dual cost $\pi_t$ of the trip coverage constraints. We generate columns in one of three ways: a labeling algorithm which is used to solve the pricing problem, a local search algorithm which approximates the solution to the pricing problem, and a local search algorithm which covers all trips at the same time. We will now discuss these in detail. \\

\paragraph{Labeling} \label{sec:labeling-evsp}
Let us consider our previously defined graph $G_{vt} = (V_{vt}, A_{vt})$. A new vehicle task can be represented as a path from $d_s$ to $d_e$, where the reduced costs of the vehicle task can be used as the length of the path. These costs can be defined as the costs incurred by driving and charging minus the sum of dual costs of the covered trips. In order to model the limited SoC range during the finding of the shortest path,  labeling algorithm algorithm can be used. 

For each node in $V$, we keep track of a set of active labels. Each of these labels contains the current costs, current SoC and the previous label. Starting in $d_e$ with an empty label, we can start propagating labels through the graph. A sketch of the labeling algorithm is outlined in Algorithm \ref{alg:evsp-labeling}.

\begin{algorithm}
\caption{E-VSP Labeling}\label{alg:evsp-labeling}
\begin{algorithmic}
\While{$|\textit{activeLabels}| > 0$} 
    \State $(label, node) \gets \textit{activeLabels.Pop()}$
    \For{$arc \in node.OutgoingArcs$}
        \State $(label', node') \gets ApplyArc(label, arc)$
        \If{$Feasible(label')$}
            \State $activeLabels.Push(label', node')$
        \EndIf
    \EndFor
\EndWhile
\end{algorithmic}
\end{algorithm}

\noindent Once all active labels have been processed, $d_e$ will now contain a list of finalized labels. We can then find a label with lowest costs and backtrack through previous label to $d_s$ in order to reconstruct the entire vehicle task. If the reduced cost of this task is negative, it will improve our solution to the relaxed E-VSP when included. 

As the amount of outgoing arcs may be large, propagating all labels can become computationally expensive. We therefore introduce a domination rule, allowing us to discard a label if a better one is known: for two labels $l$ and $l'$ on the same node $v$, we consider $l$ to be dominated by $l'$ if the costs of $l'$ are not higher and the SoC of $l'$ is not lower. If a label $l$ is dominated after the introduction of a new label $l'$, the label $l$ may be removed from the list of active nodes, reducing the total amount of considered nodes.

In order to make this domination rule stronger, SoC values may be binned into discrete buckets. This allows for small SoC differences to be discarded during the domination check, allowing more labels to become dominated and further reducing computational load. During our testing, actual SoC values were rounded down to the nearest whole percentage point for domination purposes, resulting in a total of 101 buckets. Major gains can also be achieved through proper processing order of active labels. If active labels are evaluated ordered in non-descending order of trip endtime, it is guaranteed that all domination rulings for a certain trip will have been applied before any labels from that trip are expanded. This greatly reduces the total amount of labels expanded.

The list of finalized labels in $d_e$ often contains more than one label with negative costs; in this case, multiple vehicle tasks with negative reduced costs can be extracted from the labeling result. In order to do this, we first sort the labels by cost in non-descending order. A maximum number of labels, which we shall call $N$ here, is then selected from this sorted list of labels. An attempt is first made to select labels whose trip covers are disjoint in order to encourage early coverage of all trips. After no more labels are available whose cover is disjoint, we simply add labels with new trip covers until we have added $N$ in total or until all labels with unique trip covers have been added. 

While this allows us to converge on a good solution for the relaxed problem quite quickly, solving the unrelaxed problem with the resulting vehicle tasks might still prove challenging without significant trip overlap. In an attempt to prevent this, a secondary set of vehicle tasks is generated. In this, the top $M \leq N$ generated tasks from the initial labeling round are selected as a base. For each of these bases, we iteratively block off some of the trips from the task, starting from the front. For each set of blocked of tasks, we then rerun the labeling algorithm; this causes a new set of labels to be generated which might still use part of the base, but are forced to use alternate trips for the parts which are blocked off. This results in vehicle tasks with new covers which might not be optimal in de relaxation, however they might help during the final solve. \\

\paragraph{Local search - Shortest Path} 
Without binning of SoC values, the labeling algorithm provides an exact solution to the pricing problem. For larger instances however, it might not be computationally feasible to generate enough columns in order to find a good solution. As an alternative, we can approximate a solution to the pricing problem using local search. In order to do this a simulated annealing algorithm is used to construct a vehicle task, where the objective is to once more to minimize the reduced costs of the task.  

Starting with an empty task, a collection of operations is repeatedly applied in order to modify it. After an operation is applied, the reduced cost of the new vehicle task is compared to that before the application. If the solution quality improves, the change is accepted; otherwise, it is accepted with a chance depending on both the difference in cost and how far along in the search we are. For more details, we refer to \citet{Kirk83}.

During this process, we ensure feasibility of 

\begin{algorithm}
\caption{E-VSP Simulated annealing}\label{alg:evsp-simulated-annealing}
\begin{algorithmic}
\State $vt \gets empty$
\While{$currIt < totalIts$} 
    \State $op \gets operations.GetRandom()$
    \State $vt' \gets op(vt)$
    \If{$\exp((vt.Cost - vt'.Cost) / T) > \textit{uniform}(0, 1)$}
      $vt \gets vt'$
    \EndIf
    \State $T \gets T * \alpha;\:currIts \gets currIts + 1$
\EndWhile
\end{algorithmic}
\end{algorithm}

The following set of operations is applied in order to modify the vehicle task $vt$: \\

\noindent\textit{Add trip:} from the set of currently unvisited trips, select a trip $t$ at random. In $vt$, validate that there is space to place $t$; if not, abort the operation. Otherwise, validate that there are arcs available in $A_{vt}$ to connect $t$ to its predecessor and successor in $vt$, and add $t$ to $vt$. \\

\noindent\textit{Remove trip:} from the set of currently visited trips, select a trip $t$ at random. Validate that there is an arc available from the predecessor of $t$ to the successor of $t$ in $A_{vt}$; if so, connect them and remove $t$. \\ 

\noindent\textit{Change arc:} In $vt$, select an sequential pair of trips $t$ and $t'$. Select a currently unused arc between $t$ and $t'$ from $A_{vt}$. \\ 


During these operations, we consider the placement of trips within the vehicle task to be a hard constraint; overlapping driving times are not allowed. We do allow the SoC to go out of bounds.
\paragraph{Local search - Global solution}
\todo{Invullen}

Batches of columns are generated until no columns with negative reduced costs can be found, or until a time limit is reached. Afterwards, constraint (\ref{form:x-integer}) is reinstated, making the model binary again. Branch-and-bound is then applied in order to find an integer solution. \\

\subsubsection{Postprocessing}
Once an integer solution is found to the E-VSP, there may still be overlap between the selected vehicle tasks. Depending on both the columns generated and the time limit given to solver, this number of duplicate trips may result in significant duplication of work. In order to reduce the costs incurred in driving the selected tasks, simple postprocessing is applied in order to completely eliminate any duplicate trips. 

In order to achieve this, we first make the assumption that for any pair of locations $l$ and $l'$ in $\mathcal{L}$, a corresponding deadhead $\mathcal{DH}$ exists. If this is not the case, temporarily extend $\mathcal{DH}$:  a minimum duration and distance driven between $l$ and $l'$ using sequence of existing deadheads and/or trips can be used as a stand-in. 

For each trip that is covered more than once, we can now select all tasks except the first which covers this trip. For each of these tasks, remove the duplicate trip from the task and replace it with a combination of a deadhead between the starting and ending location of a trip and idle time. If this multiple deadheads occur without a previously planned charging operation or trip between them, combine the deadheads into one by going directly from the starting location of the first deadhead to the ending location of the last deadhead, padding with idle time where needed. This will ensure that no more SoC is used and distance is driven than in the original task, and in most cases less crew time is required in order to complete the task.  

\subsubsection{Crew considerations}
Creation of vehicle tasks as described may not always result in a task which can be executed in real life. Crew members, as discussed in more detail in both Section \ref{sec:problem_def} and \ref{sec:csp-method}, are subject to a different set of constraints than the vehicles themselves. Most notably, crew members under most common labor regulations have a limit on continuous driving time. If a vehicle task contains a set of continuous trips without break or handover time which is longer than this limit, this immediately implies that this vehicle task can never be driven, and is therefore not feasible.  

Another problem arises when considering crew signing on and off for the day. Usually, the depot and maybe a select number of other locations can be used in order for a crew member to start and end their shift. Depending on the instance, a vehicle may not visit these places often enough for crew members to get on and off throughout the day, resulting in blocks that can never be driven. It can therefore be beneficial to ensure that a vehicle visits one of these crew hubs at least once in every unit. \todo{afmaken}

This problem can be resolved in one of two ways: adjusting the results of the E-VSP during the crew scheduling phase (notably something that commercial solvers such as Hastus do), or adding these crew constraints to the feasibility of the vehicle task. We selected the latter of the two options: adding additional constraints to our feasibility check during the generation of our vehicle task set. Note that for both labeling as wel as local search based approaches, this was made a hard constraint.

\subsection{CSP} \label{sec:csp-method}
For the CSP, a similar approach is taken to that of the E-VSP: (truncated) column generation of a set covering ILP. The set to be covered is now the blocks present in the selected vehicle tasks. As before, this process is broken up into steps: 
\begin{itemize}
  \item \textit{Preprocessing}, in which vehicle tasks are transformed into a set of blocks, and these blocks are in turn used to generate a graph. 
  \item \textit{Solving}, in which truncated column generation using two sources is applied in order to solve the problem of finding a set of crew duties which cover all blocks. 
\end{itemize} 

\subsubsection{Preprocessing}
In order to assign crew members to vehicles, the previously generated vehicle tasks must first be broken up into blocks. As defined previously, a block is part of a vehicle task which must be driven by a single driver; there is no possibility of handover between drivers during the block. Let $VT$ be the collection of vehicle tasks that was previously selected. For each vehicle task $vt \in VT$, let $B_{vt}$ be the set of blocks that are contained in $vt$. The total collection of to be covered blocks can then be defined as the union of all $B_{vt}$. 

Next, let us define a graph $G_{cd} = (V_{cd}, A_{cd})$ in which paths will represent crew duties. Let $V_{cd} = \{ s, e \} \cup B$, where $s$ and $e$ represent the start and end of a duty respectively. For every block $b \in B$ whose starting location is a crew base, add a \textit{sign-on arc} from $s$ to $b$. If the end location of $b$ is a crew base, add a \textit{sign-off arc} between $b$ and $e$ and $e$. Next, between each pair of blocks $b$ and $b'$, attempt to add an arc if the ending location of $b$ is the same as the starting location of $b'$ and the ending time of $b$ is not later than the starting time of $b'$. The type of the arc depends on the ending location of $b$:
\begin{itemize}
  \item If the ending location of $b$ has a place for crew members to rest and the amount of idle time is a feasible to be a break, add a \textit{break arc}.
  \item If the location is a crew base and has an idle time greater than some threshold $\nu$, add a \textit{long idle arc}. Here, the crew member signs off at the beginning of the idle time, before signing on to drive the next block. 
  \item Otherwise, add a \textit{short idle arc}, in which the crew member remains in their vehicle. 
\end{itemize}

As with $A_{vt}$, the amount of arcs $A_{cd}$ can be significantly reduced if we apply some domain knowledge in order to determine arcs that may be part of a good quality solution. In our case, this was done by filtering on a reasonable length for each arc type. The following filters were used: \todo{formatting}
\begin{itemize}
  \item Sign-on/off: Any
  \item Break arcs: [15min, 60min]
  \item Short idle arcs: [0min, 30min]
  \item Long idle arcs: [1.5h, 5h]
\end{itemize} 
Any feasible crew duty can now be represented as a path between $s$ and $e$ in $G_{cd}$. We again note that not every path from $s$ to $e$ represents a feasible duty; depending on local labor regulations and duty type constraints, differing combinations of duty length, break times and idle times might be feasible.   
\subsubsection{Solving}
Similarly to our model for the E-VSP, let us define a set covering ILP. In this, let $\textit{CD}$ be the available crew duties, let $y_j$ be a binary decision variable indicating the usage of $cd_j \in \textit{CD}$, and let $m_{j,b}$ be a parameter indicating that crew duty $cd_j$ covers block $b \in B$. Additionally, let $d_j$ be a parameter containing the total working duration of duty $cd_j$, and let $\rho_{j,\textit{broken}}$ and $\rho_{j,\textit{between}}$ be parameters indicating that $cd_j$ is of duty type \textit{broken} or \textit{between} respectively.
\begin{align}
\min \quad
& \sum_{cd_j \in \textit{CD}} y_{j} \cdot cost(cd_j)
\end{align}
Subject to:
\begin{align}
\sum_{cd_j \in \textit{CD}} y_{j}m_{j,b} &\geq 1 && \forall b \in B \label{form:block-cover}\\
\sum_{cd_j \in \textit{CD}} y_{j}(\frac{d_{j}}{\textit{8 hours}} - 1) &\leq 0 && \label{form:max-avg-shift-length}\\
\sum_{cd_j \in \textit{CD}} y_{j}*(0.3 - \rho_{j,\textit{broken}}) &\geq 0 && \label{form:max-broken-shifts}\\
\sum_{cd_j \in \textit{CD}} y_{j}*(0.1 - \rho_{j,\textit{between}}) &\geq 0 && \label{form:max-between-shifts}\\
y_{j} &\in \{ 0, 1 \} && \forall cd_j \in \textit{CD} \label{form:y-integer}
\end{align}
Here, constraint (\ref{form:block-cover}) ensures that all blocks are covered, constraint (\ref{form:max-avg-shift-length}) ensures a maximum average duty length and constraints (\ref{form:max-broken-shifts}) and (\ref{form:max-between-shifts}) ensure that a maximum of 30\% of selected duties are of type broken and 10\% are of type between respectively. This problem can be relaxed to an LP by replacing constraint (\ref{form:y-integer}) with the following:
\begin{align}
0 &\leq y_{j} \leq 1 && \forall cd_j \in \textit{CD}
\end{align}
In this relaxed problem, let the dual cost of constraints (\ref{form:block-cover}-\ref{form:max-between-shifts}) be $\tau_{\textit{cover},b}, \tau_{\textit{dur}}, \tau_{\textit{broken}}$ and $\tau_{\textit{between}}$ respectively.

The LP relaxation can once again initially be solved with dummy columns containing one block each. Note that not all of these dummy columns will be feasible, as not all blocks will start and end at a crew base; we therefore assign a penalty for the use of these dummy duties, guaranteeing that it will be cheaper to incorporate the block in a shift when possible. After the initial solution, column generation is applied using two methods: labeling in order to solve the pricing problem, and a local search to cover all blocks at once. Of these, only our labeling algorithm can make use of the dual costs $\tau$.

\paragraph{Labeling}
Let us consider our previously defined graph $G_{cd} = (V_{cd}, A_{cd})$. As with the labeling algorithm described in the E-VSP section, we can once again attempt to find a path of minimum cost through our graph $G_{cd}$ in order to generate a crew duty with minimum reduced cost. For a more detailed overview of the algorithm, we refer to Section \ref{sec:labeling-evsp}, however we will discuss details specific to the crew duty variant here. 

First, let us define our reduced costs; for a new crew duty $cd_0$, our reduced costs can be defined as 
\begin{align}
& \text{cost}(cd_0) 
- \sum_{b \in B} m_{0,b} \tau_{\textit{cover},b} 
- \left( \frac{d_0}{\text{8 hours}} \right) \tau_{\textit{dur}}  
- \tau_{\textit{broken}} (0.3 - \rho_{0,\textit{broken}}) 
- \tau_{\textit{between}} (0.1 - \rho_{0,\textit{between}})
\end{align}

Using this, we can now start with a set empty labels in $s$, one for each available duty type. During propagation, in each label keep track of:
\begin{itemize}
  \item The duty type of the label.
  \item The previous label.
  \item The starting time of the duty, defined by the first duty after $s$.
  \item The breaks arcs taken in this duty.
  \item The long idle arcs taken in this duty.
\end{itemize}
Clear domination rules are not present; the accrued reduced costs are only dominated if start time, breaks and idles are identical. As this is unlikely, removing as many labels as possible through feasibility constraints remains as the best option to ensure that computational efficiency is maintained.

Checking for feasibility during propagation can however only be done for a subset of the provided labor regulation rules. Properties such as continuous driving time, maximum end time and use of long idle arcs can already be used in order to remove labels which cannot result in a feasible duty of a given type. Before reaching $e$ however, the end time and distribution of breaks the in duty are not yet finalized. This notably implies that it is not possible to check for sufficient breaks while propagation is still in progress: suppose some label in the process indicates that its corresponding duty will have a length of at least 6 hours, requiring a total of 40 minutes of break time under Dutch labor regulations. If the current amount of break time is 20 minutes, this may still become feasible before reaching $e$ by taking another break arc; we can therefore not discard this label before we know when the duty will end.  

Once propagation has completed, the duties described by the labels in $e$ can be finalized. Only then can we do our full feasibility check, discarding any labels which are not valid given their state in $e$. As with the E-VSP, what remains is a list of labels which describe feasible duties with negative reduced cost. We will then take at most $N$ of these duties, prioritizing block-disjoint duties. Next, secondary columns are generated for the best $M \leq N$ of the found duties. After this, the complete set of duties is made available to the LP relaxation, before repeating the process with new values of $\tau$. 
\printbibliography
\end{document}
